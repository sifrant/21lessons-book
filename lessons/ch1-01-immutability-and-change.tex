\chapter{Immutability and Change}
\label{les:1}

\begin{chapquote}{Alice}
\enquote{Copak se se mnou v noci stala nějaká změna? Počkat: byla jsem to já, 
když jsem ráno vstávala? Tak se mi zdá, že mi bylo nějak divně. Ale jestli 
to nejsem já, ptám se dál, kdo tedy jsem? To je ta záhada!}
\end{chapquote}

Bitcoin je ze své podstaty těžké popsat. Je to nová věc a jakýkoli pokus 
o srovnání s předchozími koncepty - ať už ho nazveme digitálním zlatem nebo 
internetem peněz - se nutně mine účinkem. Ať už je vaše oblíbené přirovnání 
jakékoli, dva aspekty Bitcoinu jsou naprosto zásadní: decentralizace a neměnnost.

\paragraph{}
Jedním ze způsobů, jak o Bitcoinu přemýšlet, je automatizovaná společenská smlouva\footnote{Hasu,
Unpacking Bitcoin's Social Contract~\cite{social-contract}}. Software je jen 
jedním z dílků skládačky a doufat, že změníte Bitcoin změnou softwaru, je marná 
snaha. Člověk by musel přesvědčit zbytek sítě, aby takové změny přijal, což je 
mnohem více psychologická záležitost než úsilí v oblasti softwarového inženýrství.

\paragraph{}
Následující věta může znít na první pohled absurdně, stejně jako mnoho jiných věcí 
v tomto prostoru, ale přesto věřím, že je hluboce pravdivá: Bitcoin nezměníte, 
ale Bitcoin změní vás.

\begin{quotation}\begin{samepage}
\enquote{Bitcoin nás změní víc než my jeho.}
\begin{flushright} -- Marty Bent\footnote{Tales From the Crypt~\cite{tftc21}}
\end{flushright}\end{samepage}\end{quotation}

Trvalo mi dlouho, než jsem si uvědomil hloubku tohoto tvrzení. Protože Bitcoin 
je jen software a vše je open-source, můžete jednoduše měnit věci podle libosti, 
ne? Omyl. \textit{Velmi} špatně. Není divu, že tvůrce Bitcoinu to věděl až příliš dobře.

\begin{quotation}\begin{samepage}
\enquote{Povaha Bitcoinu je taková, že jakmile byla vydána verze 0.1, byl základní 
design vytesán do kamene po zbytek jeho existence.}
\begin{flushright} -- Satoshi Nakamoto\footnote{BitcoinTalk forum post: `Re:
Transactions and Scripts\ldots'~\cite{satoshi-set-in-stone}}
\end{flushright}\end{samepage}\end{quotation}

Mnoho lidí se pokoušelo povahu Bitcoinu změnit. Doposud nikdo z nich neuspěl. 
Zatímco existuje nepřeberná řada forků a altcoinů, Bitcoinová síť stále funguje 
stejně jako v době, kdy byl spuštěn první uzel. Na altcoinech z dlouhodobého 
hlediska nezáleží. Forky nakonec vyhladoví k smrti. Důležitý je Bitcoin. Dokud 
se nezmění naše základní chápání matematiky a/nebo fyziky, bude Bitcoinovému 
medojedovi i nadále všechno jedno.

\begin{quotation}\begin{samepage}
\enquote{Bitcoin je prvním případem nové formy života. Žije a dýchá na internetu. 
Žije proto, že může platit lidem za to, že ho udržují při životě. [\ldots] Nelze 
ho změnit. Nelze s ním polemizovat. Nelze s ním manipulovat. Nelze ho zpochybnit. 
Nelze jej zastavit. [\ldots] Kdyby jaderná válka zničila polovinu naší planety, 
bude žít dál, nepoškozený.}
\begin{flushright} -- Ralph Merkle\footnote{DAOs, Democracy and
Governance,~\cite{merkle-dao}}
\end{flushright}\end{samepage}\end{quotation}

Tlukot srdce Bitcoinové sítě přežije nás všechny.

~

Uvědomění si výše uvedeného mě změnilo mnohem víc, než mě kdy změní uplynulé bloky 
Bitcoinového blockchainu. Změnilo to mé časové preference, mé chápání ekonomie, 
mé politické názory a mnoho dalšího. K čertu, dokonce to mění i lidský 
jídelníček\footnote{Inside the World of the Bitcoin Carnivores,~\cite{carnivores}}. 
Pokud vám to všechno zní šíleně, jste v dobré společnosti. 
Všechno je to šílené, a přesto se to děje.

~

\paragraph{Bitcoin mě naučil, že se nezmění. Já ano.}

% ---
%
% #### Through the Looking-Glass
%
% - [Bitcoin's Gravity: How idea-value feedback loops are pulling people in][gravity]
% - [Lesson 18: Move slowly and don't break things][lesson18]
%
% #### Down the Rabbit Hole
%
% - [Unpacking Bitcoin's Social Contract][automated social contract]: A framework for skeptics by Hasu
% - [DAOs, Democracy and Governance][Ralph Merkle] by Ralph C. Merkle
% - [Marty's Bent][bent]: A daily newsletter highlighting signal in Bitcoin by Marty Bent
% - [Technical Discussion on Bitcoin's Transactions and Scripts][Satoshi Nakamoto] by Satoshi Nakamoto, Gavin Andresen, and others
% - [Inside the World of the Bitcoin Carnivores][carnivores]: Why a small community of Bitcoin users is eating meat exclusively by Jordan Pearson
% - [Tales From the Crypt][tftc] hosted by Marty Bent
%
% <!-- Internal -->
% [gravity]: 
% [lesson18]: {{ 'bitcoin/lessons/ch3-18-move-slowly-and-dont-break-things' | absolute_url }}
%
% <!-- Further Reading -->
% [automated social contract]: https://medium.com/@hasufly/bitcoins-social-contract-1f8b05ee24a9
% [carnivores]: https://motherboard.vice.com/en_us/article/ne74nw/inside-the-world-of-the-bitcoin-carnivores
% [tftc]: https://tftc.io/tales-from-the-crypt/
% [bent]: https://tftc.io/martys-bent/
%
% <!-- Quotes -->
% [Ralph Merkle]: http://merkle.com/papers/DAOdemocracyDraft.pdf
% [Satoshi Nakamoto]: https://bitcointalk.org/index.php?topic=195.msg1611#msg1611
%
% <!-- Twitter People -->
% [Marty Bent]: https://twitter.com/martybent
%
% <!-- Wikipedia -->
% [alice]: https://en.wikipedia.org/wiki/Alice%27s_Adventures_in_Wonderland
% [carroll]: https://en.wikipedia.org/wiki/Lewis_Carroll
