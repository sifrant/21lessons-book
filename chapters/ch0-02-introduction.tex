\chapter*{Introduction}
\label{ch:introduction}

\begin{chapquote}{Lewis Carroll, \textit{Alice in Wonderland}}
\enquote{Ale já nechci mezi potrhlíky,} bránila se Alenka. \enquote{Málo platné,} 
řekla kočka, \enquote{tady jsme všichni potrhlí. Já jsem potrhlá, ty jsi taky 
potrhlá.} \enquote{Jak to víš, že jsem potrhlá?} zeptala se Alenka. \enquote{To je 
jisté, jinak bys sem nechodila.}\end{chapquote}

V říjnu 2018 položil Arjun Balaji nevinnou otázku: 
\textit{What have you learned from Bitcoin?} Po žalostně neúspěšném pokusu 
o odpověď krátkým tweetem, mi došlo, že věcí, které jsem se naučil, je příliš 
mnoho na to, abych na ně mohl odpovědět rychle. Pokud vůbec.

To, co jsem se naučil, se samozřejmě týká Bitcoinu - nebo s ním alespoň souvisí. 
Nicméně, i když jsou zde některé z vnitřních mechanismů Bitcoinu objasněny, 
následující lekce nejsou vysvětlením toho, jak Bitcoin funguje nebo co to je. 
Mohly by však pomoci prozkoumat některé ze záležitostí, jichž se Bitcoin dotýká: 
filosofické otázky, ekonomické skutečnosti a technologické inovace.

\begin{center}
  \includegraphics[width=7cm]{assets/images/the-tweet.png}
\end{center}

\textit{Jednadvacet lekcí} je strukturováno do svazků po sedmi, což vyúsťuje 
ve tři kapitoly. Každá kapitola se na Bitcoin dívá jinou optikou a extrahuje 
z něj poučení, která lze získat při zkoumání této podivné sítě z různých 
úhlů pohledu.

\paragraph{\hyperref[ch:philosophy]{Kapitola I}}{zkoumá filosofické učení 
Bitcoinu. Souhru neměnnosti a změny, koncept skutečné vzácnosti, neposkvrněné 
početí Bitcoinu, problém identity, rozpor replikace a polohy, sílu svobody 
slova a hranice poznání.
}

\paragraph{\hyperref[ch:economics]{Kapitola II}}{se zabývá ekonomickým poznáním 
Bitcoinu. Lekce o finanční negramotnosti, inflaci, hodnotě, penězích a jejich 
historii, bankovnictví frakčních rezerv a o tom, jak Bitcoin mazaně a oklikou 
znovu zavádí zdravé peníze.}

\paragraph{\hyperref[ch:technology]{Kapitola III}}{se věnuje některým poznatkům 
získaným při zkoumání technologie Bitcoinu. Proč je v číslech síla, úvahy o důvěře, 
proč určování času vyžaduje práci, jak je pomalý vývoj a nenarušování systému 
předností, a nikoli chybou, co nám může vznik Bitcoinu říci o soukromí, proč 
cypherpunkové píší kód (a ne zákony) a jaké metafory by mohly být užitečné 
pro zkoumání budoucnosti Bitcoinu.}

~

Každá lekce obsahuje v textu několik citátů a odkazů. Prozkoumal-li jsem nějakou 
myšlenku podrobněji, odkazy na mé související práce najdete v části "Za zrcadlem". 
Pokud byste rádi šli více pod povrch, odkazy na nejdůležitější materiály jsou 
uvedeny v části "Hlouběji králičí norou". Obojí najdete na konci každé lekce. 
(V případě, že jsem nalezl vhodný a dostatečně kvalitní článek v češtině, odkazuji 
na něj. Jinak ponechávám odkazy na původní anglické zdroje - pozn. překladatele)

Přestože jsou určité předchozí znalosti o Bitcoinu přínosné, doufám, že tyto lekce 
zvládne strávit každý zvídavý čtenář. I když některé z lekcí spolu souvisejí, 
každá by měla být schopna obstát sama o sobě a lze je číst samostatně. Technickému 
žargonu jsem se snažil maximálně vyhýbat, avšak některým slovíčkům, specifickým 
pro danou oblast, se nelze zcela vyvarovat.

Doufám, že můj text poslouží ostatním jako inspirace, aby se ponořili pod povrch 
a prozkoumali některé z hlubších otázek, které Bitcoin vyvolává. Mou vlastní 
inspiraci mi poskytlo množství jiných autorů a jim všem jsem neskonale vděčný.

V neposlední řadě: cílem mého psaní není vás o čemkoli přesvědčovat. Mým cílem 
je přimět vás k zamyšlení a ukázat vám, že Bitcoin je mnohem víc, než se na první 
pohled zdá. Nemohu vám ani říct, co Bitcoin je, nebo co vás naučí. To si budete 
muset zjistit sami pro sebe.

\begin{quotation}\begin{samepage}
\enquote{Z týhle cesty se nemůžeš vrátit. S modrou kapslí všechno skončí, probudíš 
se doma a budeš věřit, čemu budeš chtít. Ale s tou červenou - zůstaneš v Říši divů 
a já ti ukážu, jak hluboko králičí nora vede.}
\begin{flushright} -- Morpheus
\end{flushright}\end{samepage}\end{quotation}

\begin{figure}
  \includegraphics{assets/images/bitcoin-orange-pill.jpg}
  \caption*{Pamatujte: Vše, co nabízím, je pravda. Nic víc.}
  \label{fig:bitcoin-orange-pill}
\end{figure}

%
% [Morpheus]: https://en.wikipedia.org/wiki/Red_pill_and_blue_pill#The_Matrix_(1999)
% [this question]: https://twitter.com/arjunblj/status/1050073234719293440
%
% <!-- Internal -->
% [chapter1]: {{ 'bitcoin/lessons/ch1-00-philosophy' | absolute_url }}
% [chapter2]: {{ 'bitcoin/lessons/ch2-00-economics' | absolute_url }}
% [chapter3]: {{ 'bitcoin/lessons/ch3-00-technology' | absolute_url }}
%
% <!-- Wikipedia -->
% [alice]: https://en.wikipedia.org/wiki/Alice%27s_Adventures_in_Wonderland
% [carroll]: https://en.wikipedia.org/wiki/Lewis_Carroll
